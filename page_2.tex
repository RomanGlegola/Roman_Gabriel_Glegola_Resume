\newpage 
%-------------------------------------------------------------
%-------------------------------------------------------------
%-------------------------------------------------------------
%                       PAGE 2
%-------------------------------------------------------------
%-------------------------------------------------------------
    \begin{leftcolumn*} \noindent \footnotesize
    %-------------------------------------------------------------
    %-------------------------------------------------------------
    %-------------------------------------------------------------
    %                       LEFT COLUMN
    %-------------------------------------------------------------
    %-------------------------------------------------------------
    \phantom{} \\ % To leave a margin with the top of the page
    {\color{white}
        %-------------------------------------------------------------
        %                       CI/CD
        %-------------------------------------------------------------
        \heading{\faWrench}{CI/CD}
        \begin{minipage}[c]{\leftcolwidth}
            \begin{tabular}{r|l}
                \href{https://learn.microsoft.com/en-us/azure/devops/get-started/?view=azure-devops}{\textbf{Azure DevOps}} 
                    & \pictofraction{4}\\[0.3em]
                \href{https://trello.com/guide}{\textbf{Trello}} 
                    & \pictofraction{4}\\[0.3em]
                \href{https://www.zabbix.com/manuals}{\textbf{Zabbix}} 
                    & \pictofraction{4}\\[0.3em]
                \href{https://confluence.atlassian.com/jira}{\textbf{Jira}} 
                    & \pictofraction{3}\\[0.3em]
                \href{https://confluence.atlassian.com/alldoc/atlassian-documentation-32243719.html}{\textbf{Confluence}} 
                    & \pictofraction{3}\\[0.3em]
                \href{https://wiki.pmease.com/display/QB13/Documentation+Home}{\textbf{QuickBuild}} 
                    & \pictofraction{3}\\[0.3em]
                \href{https://www.jenkins.io/doc/}{\textbf{Jenkins}} 
                    & \pictofraction{3}\\[0.3em]
                \href{https://support.testrail.com/hc/en-us/articles/7076810203028-Introduction-to-TestRail}{\textbf{TestRail}} 
                    & \pictofraction{1}\\[0.3em]
            \end{tabular}
        \end{minipage} % \CI/CD
    } % \color{white}
    
    {\color{white}
        %-------------------------------------------------------------
        %                       OPERATING SYSTEMS
        %-------------------------------------------------------------
        \heading{\faWrench}{OS \& WM}
        \begin{minipage}[c]{\leftcolwidth}
            \begin{tabular}{r|l}
                \href{https://wiki.archlinux.org/}{\textbf{Archlinux}} 
                    & \pictofraction{4}\\[0.3em]
                \href{https://learn.microsoft.com/en-us/windows/}{\textbf{Windows}} 
                    & \pictofraction{4}\\[0.3em]
                \href{https://www.kernel.org/doc/html/latest/}{\textbf{Linux}} 
                    & \pictofraction{3}\\[0.3em]
                \href{https://docs.ubuntu.com/}{\textbf{Ubuntu}} 
                    & \pictofraction{3}\\[0.3em]
                \href{https://learn.microsoft.com/en-us/windows/wsl/}{\textbf{WSL}} 
                    & \pictofraction{2}\\[0.3em]
                \href{https://learn.microsoft.com/en-us/troubleshoot/windows-server/system-management-components/what-is-microsoft-management-console}{\textbf{MMC}} 
                    & \pictofraction{2}\\[0.3em]
                \href{https://docs.docker.com/}{\textbf{Docker}} 
                    & \pictofraction{2}\\[0.3em]
                \href{https://docs.vmware.com/}{\textbf{VMWare}} 
                    & \pictofraction{2}\\[0.3em]
                \href{https://www.virtualbox.org/wiki/Documentation}{\textbf{Virtual Box}} 
                    & \pictofraction{2}\\[0.3em]
            \end{tabular}
        \end{minipage} % \OPERATING SYSTEMS
    } % \color{white}
    \end{leftcolumn*}
    \begin{rightcolumn}\noindent \small
    %-------------------------------------------------------------
    %-------------------------------------------------------------
    %                       RIGHT COLUMN
    %-------------------------------------------------------------
    %-------------------------------------------------------------
    \phantom{} \\ % To leave a margin with the top of the page
        %-------------------------------------------------------------
        %                     FORMAL-EDUCATION
        %-------------------------------------------------------------
        \heading{\faGraduationCap}{Formal Education}
        % MERITO
        \cvevent{2021}{Current}{Computer Science: Computer Networks}{University MERITO in Gdańsk}{Gdańsk, Poland}{assets/logo-merito.png}
            {I undertook \href{https://www.merito.pl/gdansk/studia-i-szkolenia/studia-i-stopnia/kierunki-i-specjalnosci/informatyka/inzynieria-sieci-komputerowych}{\textbf{Computer Science studies in Computer Networks}} at the \href{https://www.merito.pl/gdansk/}{\textbf{School of Banking in Gdańsk}} during my first year of work at SolwIT with the aim of furthering my development in the field of IT in the DevOps, SysAdmin, Security, and similar roles related to computer network infrastructure.}
        \vspace{\itemspace}\\
        % WSB
        \cvevent{2016}{2018}{Law in Business}{School of Banking in Gdańsk}{Gdańsk, Poland}{assets/logo-wsb.jpg}
            {I embarked on \href{https://www.merito.pl/gdansk/studia-i-szkolenia/studia-i-stopnia/kierunki-i-specjalnosci/prawo-w-biznesie/prawo-i-zarzadzanie}{\textbf{Business Law}} studies at the \href{https://www.merito.pl/gdansk/}{\textbf{School of Banking in Gdańsk}} immediately after completing my military service. The motivation for this decision was, in my opinion, a desire for more knowledge beyond what I had gained from my previous field of study. The duration of my studies was shortened by one semester due to my request to combine the last two semesters into one.}
        \vspace{\itemspace}\\
        % PWSZ
        \cvevent{2008}{2013}{Economic Managment}{State Higher Vocational School in Elbląg}{Elbląg, Poland}{assets/logo-pwsz.jpeg}
            {Immediately after high school, I began studying \href{https://ans-elblag.pl/studia-ekonomia-licencjackie.html}{\textbf{Managerial Economics}} at the \href{https://ans-elblag.pl/}{\textbf{State Higher Vocational School in Elbląg}}S. Due to the difficulty I faced with the exact sciences, I was compelled to repeat a semester.}
        \vspace{\sectionspace}
    \end{rightcolumn} % \FORMAL-EDUCATION
    \begin{leftcolumn*}\noindent \footnotesize
    %-------------------------------------------------------------
    %-------------------------------------------------------------
    %                       LEFT COLUMN
    %-------------------------------------------------------------
    %-------------------------------------------------------------
    {\color{white}
        %-------------------------------------------------------------
        %                       PHILOSOPHY
        %-------------------------------------------------------------
        \hspace{-2.4pt}\heading{\faQuoteLeft}{Philosophy}
        \fbox{
            \begin{minipage}[l]{0.9\leftcolwidth}
                Here are some thoughts that guide my\\actions as a IT Specialist.\\[1em]
                    \simplequote{The question isn't who is going to let me; it's who is going to stop me.}
                    {\href{https://en.wikipedia.org/wiki/Ayn_Rand}{\textbf{Ayn Rand}}}
                \vspace{\itemspace}\\
                    \simplequote{One of the great mistakes is to judge policies and programs by their intentions rather than their results.}
                    {\href{https://en.wikipedia.org/wiki/Milton_Friedman}{\textbf{Milton Friedman}}}
                \vspace{\itemspace}\\
                    \simplequote{I don’t care if it works on your machine! We are not shipping your machine!}
                    {\href{https://www.linkedin.com/in/ovidiupl/}{\textbf{Ovidiu Platon}}}
                \vspace{\itemspace}\\
                    \simplequote{I have this hope that there is a better way. Higher-level tools that actually let you see the structure of the software more clearly will be of tremendous value.}
                    {\href{https://www.linkedin.com/in/guido-van-rossum-4a0756/}{\textbf{Guido van Rossum}}}
                \vspace{\itemspace}\\
                    \simplequote{Programming is about managing complexity: the complexity of the problem, laid upon the complexity of the machine. Because of this complexity, most of our programming projects fail.}
                    {\href{https://www.linkedin.com/in/bruceeckel/}{\textbf{Bruce Eckel}}}
                \vspace{\itemspace}\\
                    \simplequote{You are not reading this book because a teacher assigned it to you, you are reading it because you have a desire to learn, and wanting to learn is the biggest advantage you can have.}
                    {\href{https://www.linkedin.com/in/calthoff/}{\textbf{Cory Althoff}}}
                \vspace{\itemspace}\\
                    \simplequote{If, at first, you do not succeed, call it version 1.0.}
                    {\href{https://www.linkedin.com/in/krrw/}{\textbf{Khayri R.R. Woulfe}}}
                \vspace{\itemspace}\\
                    \simplequote{Dress warmly, eat healthy.}
                    {\href{https://archiwum2019-10bkpanc.wp.mil.pl/pl/11.html}{\textbf{Col. Artur Pikoń}}}
                \vspace{\itemspace}\\
                    \simplequote{My beloved son.}{My mom (ha ha)}
            \end{minipage} % \PHILOSOPHY
        }%\fbox
    } % \color{white}
    \vspace{\itemspace}\\ 
    {\color{white} 
        %-------------------------------------------------------------
        %                       LANGUAGES
        %-------------------------------------------------------------
        \phantom{} \\ % To leave a margin with the top of the page
        \heading{\faGlobe}{Languages}
        \begin{minipage}[r]{\leftcolwidth}
            \begin{tabular}{r|l}
                English & Working knowledge\\[0.3em]
                Polish & Native\\[0.3em]
                German & Notions\\[0.3em]
                Russian & Notions
            \end{tabular}
        \end{minipage} % \LANGUAGES
    } % \color{white}
    \end{leftcolumn*}
    \begin{rightcolumn}\noindent \small
    %-------------------------------------------------------------
    %-------------------------------------------------------------
    %                       RIGHT COLUMN
    %-------------------------------------------------------------
    %-------------------------------------------------------------
        %-------------------------------------------------------------
        %                       SELF-EDUCATION
        %-------------------------------------------------------------
        \hspace{-2.4pt}\heading{\faTv}{Self-Education}
            % Usage: \onlinecourse{1:date}{2:course title}{3:organisation-name}{4:organisation-logo}{5:text}{6:certificates/results}
        % AMBERTEAM
        \onlinecourse{2023}{ISTQB Certified Tester Advanced - Technical Test Analyst}{AmberTeam Testing sp. z o.o.}{assets/logo-amber.png}
            {During this multi-day training provided by \href{https://www.amberteam.pl/}{\textbf{AmberTeam Testing}}, I gained knowledge in the field of \href{https://www.istqb.org/certifications/technical-test-analyst}{\textbf{ISTQB Technical Test Analyst}}, learned about the technical aspects of risk analysis, static code analysis and other advanced aspects of the software testing process.}
        {Results: \href{https://drive.google.com/file/d/1WwfEmnIlJKEFpINr4jTWSFmnCXFLiDe3/view}{\textbf{Statement of Attendance}}.}\\
        \vspace{\itemspace}\\
        % IKU PCAP
        \onlinecourse{2023}{PCAP: Programming Essentials in Python}{IKU LTD}{assets/logo-iku.jpeg}
            {During this \href{https://www.python.org/}{\textbf{Python}} course organised by \href{http://iku-szkolenia.pl/}{\textbf{Instytut Kształcenia Ustawicznego}}, we prepared for the supervised \href{https://www.netacad.com/courses/programming/pcap-programming-essentials-python}{\textbf{PCAP: Programming Essentials in Python}} certification exam, under the auspices of the \href{https://pythoninstitute.org/}{\textbf{Python Institute}} as part of the \href{https://www.netacad.com/}{\textbf{CISCO Academy}}.}
        {Results: \href{https://drive.google.com/file/d/19MBRMRG7BXdnB-8vFiySWSYVE58n8XDh/view}{\textbf{Statement of Attendance}}.}\\
        \vspace{\itemspace}\\   
        % IKU Angular
        \onlinecourse{2023}{Angular - Building Modern and Efficient Applications}{IKU LTD}{assets/logo-iku.jpeg}
            {During this week-long training in building web applications with \href{https://angular.io/}{\textbf{Angular}}, I learned the syntax and grammar of the \href{https://www.typescriptlang.org/}{\textbf{TypeScript}} language and got acquainted with the \href{https://developer.android.com/studio}{\textbf{Android Studio}} tool. The training also covered \href{https://developer.mozilla.org/en-US/docs/Web/CSS}{\textbf{CSS}} and \href{https://www.w3.org/}{\textbf{HTML}} templates for creating web apps.}
        {Results: \href{https://drive.google.com/file/d/1Q6IBp6MqAddaCwaZqZxMo0kMkV-l9WKb/view}{\textbf{Statement of Attendance}}.}\\
        \vspace{\itemspace}\\
        % NETACAD
        \onlinecourse{2022}{CCNA: Introduction to Networks}{Buisiness University in Gdańsk}{assets/logo-netacad.jpg}
            {The \href{https://www.netacad.com/}{\textbf{Netacad Academy}} course involved configuring switches and endpoint devices for network resource access. During this course I gained knowledge of Ethernet operation in switched networks, configuring routers for remote connectivity, creating IPv4 and IPv6 addressing schemes, supporting network applications via the OSI model, establishing a secure small network, and troubleshooting small network issues.}
        {Results: \href{https://drive.google.com/file/d/1vHRxM6J5HI5tNsH_mfFKZNjf9m25LXc9/view}{\textbf{Statement of Attendance}}.}\\
        \vspace{\itemspace}\\
        % INFOSHARE
        \onlinecourse{2019}{Python from Scratch}{InfoShare Academy sp. z o.o.}{assets/logo-infoshare.jpg}
            {The course from \href{https://infoshareacademy.com/}{\textbf{Infoshare Academy}} in 45 hours instilled in me the basic knowledge of \href{https://www.python.org/}{\textbf{Python}} programming language syntax and the fundamentals of \href{https://www.djangoproject.com/}{\textbf{Django}} and \href{https://flask.palletsprojects.com/}{\textbf{Flask}} frameworks. I believe that the money spent on this course has paid off many times over.}
        {Results: \href{https://drive.google.com/file/d/10PP3qx6BOw9ZF7lJsH-YOfFBLjtf5vl8/view}{\textbf{Statement of Attendance}}.}\\
        \vspace{\itemspace}\\
        % SDA
        \onlinecourse{2019}{Manual Tester}{SDA sp. z o.o.}{assets/logo-sda.jpg}
            {The course from \href{https://sdacademy.dev/}{\textbf{Software Development Academy}} was my leap of faith towards entering the IT field. The course aimed to prepare for the \href{https://www.istqb.org/certifications/certified-tester-foundation-level}{\textbf{ISTQB FL}} exam and also to introduce various tools used in testing such as \href{https://www.java.com/}{\textbf{Java}}, \href{https://maven.apache.org/what-is-maven.html}{\textbf{Maven}}, \href{https://www.selenium.dev/}{\textbf{Java Selenium WebDriver}}, \href{https://cucumber.io/}{\textbf{Cucumber}}, \href{https://www.postman.com/}{\textbf{Postman}}, \href{https://jmeter.apache.org/}{\textbf{JMeter}}, \href{https://www.mysql.com/}{\textbf{MySQL}}, \href{https://www.testrail.com/}{\textbf{TestRail}}, \href{https://sdacademy.dev/} and services from \href{https://www.atlassian.com/}{\textbf{Atlassian}}.}
        {Results: \href{https://drive.google.com/file/d/1KNqjtyxgEgl3oX9MsnACIOpoZYgP0HpX/view}{\textbf{Statement of Attendance}}.}\\
        \vspace{\itemspace}\\
    \end{rightcolumn} % \SELF-EDUCATION